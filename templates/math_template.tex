\documentclass[11pt, a4paper]{article} % Consider 'amsart' for publishing

% ===== ESSENTIAL PACKAGES =====
\usepackage{amsmath, amssymb, amsthm} % AMS mathematical facilities
\usepackage{mathtools} % For advanced math formatting (e.g., cases)
\usepackage{geometry} % For adjustable margins
\geometry{a4paper, margin=1in} % Set margins to 1 inch

% ===== CUSTOM THEOREM ENVIRONMENTS =====
\newtheorem{theorem}{Theorem}[section]
\newtheorem{lemma}[theorem]{Lemma}
\newtheorem{proposition}[theorem]{Proposition}
\newtheorem{corollary}[theorem]{Corollary}
\newtheorem{definition}[theorem]{Definition}
\newtheorem{example}[theorem]{Example}
\newtheorem{remark}[theorem]{Remark}

% ===== SHORTCUT COMMANDS =====
% Common sets (mathbb)
\newcommand{\R}{\mathbb{R}} % Real numbers
\newcommand{\N}{\mathbb{N}} % Natural numbers
\newcommand{\Z}{\mathbb{Z}} % Integers
\newcommand{\C}{\mathbb{C}} % Complex numbers
\newcommand{\Q}{\mathbb{Q}} % Rational numbers

% Common operators (mathrm)
\newcommand{\diff}{\mathop{}\!\mathrm{d}} % For integrals: dx -> \diff x
\DeclareMathOperator{\arcsec}{arcsec} % Declare operators that aren't predefined

% ===== DOCUMENT BEGIN =====
\title{Title of Your Mathematical Work}
\author{Your Name}
\date{\today} % You can remove this for a fixed date or leave blank

\begin{document}
\maketitle

\begin{abstract}
    A concise summary of your paper's purpose and main results.
\end{abstract}

\section{Introduction}
\label{sec:introduction}

This is where you introduce the problem. You can cite equations like \eqref{eq:favorite} or theorems like \ref{thm:main}.

\section{Main Results}
\label{sec:main_results}

\begin{definition}
    A function \( f: \R \to \R \) is called \emph{interesting} if it satisfies the following condition:
    \[
        \int_{-\infty}^{\infty} |f(x)|^2 \diff x < \infty.
    \]
\end{definition}

\begin{theorem}[Main Theorem]
\label{thm:main}
    Every interesting function \( f \) is continuous almost everywhere.
\end{theorem}

\begin{proof}
    The proof involves a standard argument using the density of continuous functions in \( L^2(\R) \). Let \( \epsilon > 0 \). Since \( f \in L^2(\R) \), there exists...
    \[
        \lim_{n \to \infty} \left( \sum_{k=1}^{n} \frac{1}{k^2} \right) = \frac{\pi^2}{6}.
    \]
    \qedhere % Use this to place the QED symbol correctly inside displayed math
\end{proof}

You can also use aligned equations:
\begin{equation}
\label{eq:favorite}
    \begin{aligned}
        e^{i\pi} + 1 &= 0, \\
        \frac{\diff}{\diff t} \left( \frac{\partial L}{\partial \dot{q}_k} \right) - \frac{\partial L}{\partial q_k} &= 0.
    \end{aligned}
\end{equation}

Or a case structure:
\[
    f(x) = \begin{dcases}
        \frac{\sin(x)}{x} & \text{if } x \neq 0, \\
        1                 & \text{if } x = 0.
    \end{dcases}
\]

\section*{Acknowledgements} % The asterisk (*) removes section numbering
Optional acknowledgements.

% ===== BIBLIOGRAPHY =====
% Uncomment the lines below if you have references
% \bibliographystyle{plain}
% \bibliography{references}

\end{document}
